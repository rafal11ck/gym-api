\documentclass[../../spr.tex]{subfiles}

\begin{document}

\section{Podsumowanie}

\subsection{Wnioski z realizacji projektu}
W ramach projektu udało się stworzyć działające i stabilne API dla aplikacji fitness.
Zostało ono oparte na architekturze MVC i pozwala na zarządzanie sesjami treningowymi,
ćwiczeniami, członkostwami użytkowników oraz obsługę płatności.
Wszystkie operacje typu CRUD (\texttt{GET}, \texttt{POST}, \texttt{PATCH},
\texttt{DELETE}) zostały poprawnie zaimplementowane w odpowiednich kontrolerach.
Dzięki podziałowi na modele, widoki i kontrolery, kod jest bardziej przejrzysty i
łatwiejszy do rozwijania.
Dodatkowo zastosowanie zasad SOLID pomogło w utrzymaniu porządku w kodzie i
jego modularności. Wprowadzenie paginacji pozwala lepiej obsługiwać większe
ilości danych.

\subsection{Ocena osiągniętych rezultatów i refleksje na temat procesu implementacji}
Zrealizowane API spełnia najważniejsze założenia projektu.
W trakcie implementacji można było zauważyć, jak ważne są bezpieczne logowanie \textit{(OAuth2)},
możliwość sortowania danych i paginacja.
Jednym z trudniejszych elementów było zapewnienie spójności danych pomiędzy
powiązanymi tabelami (np. ćwiczenia i uczestnicy).
Problem udało się rozwiązać poprzez dobrze zaprojektowane struktury żądań i
odpowiedzi oraz ich walidację.

\subsection{Propozycje usprawnień lub dalszego rozwoju aplikacji}
W przyszłości można by dodać możliwość śledzenia treningów w czasie rzeczywistym,
co uatrakcyjniłoby korzystanie z aplikacji. Warto też pomyśleć o takich funkcjach
jak powiadomienia, historia aktywności użytkownika czy integracja z urządzeniami
fitness (np. smartwatche).Dodatkowym usprawnieniem mogłoby być
wprowadzenie systemu rekomendacji treningów dopasowanych do celów użytkownika
(np. redukcja, masa, kondycja).
Rozważenie wsparcia dla wielu języków mogłoby rozszerzyć grono odbiorców aplikacji.
\end{document}

