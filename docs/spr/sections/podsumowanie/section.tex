\documentclass[../../spr.tex]{subfiles}

\begin{document}

\section{Podsumowanie}

\subsection{Wnioski z realizacji projektu}
Możemy stwierdzić, że stworzono solidne API dla aplikacji fitness,
oparte na architekturze MVC, umożliwiające zarządzanie sesjami treningowymi,
ćwiczeniami, członkostwami użytkowników oraz przetwarzanie płatności.
Zaimplementowanie operacji CRUD (GET, POST, PATCH, DELETE) w różnych kontrolerach
w ramach wzorca MVC pokazuje kompleksowe rozwiązanie backendowe,
gdzie modele obsługują logikę biznesową, widoki prezentują dane,
a kontrolery zarządzają przepływem. Użycie zasad SOLID
zapewniło modułowość i łatwość rozbudowy kodu.
Ponadto zastosowanie paginacji zwiększa skalowalność i efektywną obsługę dużych ilości danych.

\subsection{Ocena osiągniętych rezultatów i refleksje na temat procesu implementacji}
Oceniając rezultaty, API spełnia podstawowe wymagania platformy fitness.
Proces podkreślił znaczenie bezpiecznego uwierzytelniania (OAuth2) oraz elastycznych opcji sortowania i paginacji.
Wyzwania obejmowały zapewnienie spójności danych między powiązanymi encjami (np. ćwiczenia i uczestnicy),
co zostało rozwiązane dzięki uporządkowanym schematom żądań i odpowiedzi.

\subsection{Propozycje usprawnień lub dalszego rozwoju aplikacji}
Proponując usprawnienia, można rozważyć dodanie śledzenia treningów w czasie rzeczywistym.
\end{document}
