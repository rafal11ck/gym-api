\documentclass[../../spr.tex]{subfiles}

\begin{document}

\section{Testy}
\subsection{Opis metod testowania (np. testy manualne i automatyczne)}

W projekcie zastosowano automatyczne testy integracyjne warstwy kontrolerów w aplikacji Spring Boot.
Testy uruchamiane są na pełnym kontekście aplikacji,
co pozwala na weryfikację poprawności działania endpointów HTTP wraz z rzeczywistą logiką biznesową.
W celu zapewnienia izolacji testów od zewnętrznych systemów, komponenty komunikujące się z
usługami zewnętrznymi są mockowane, natomiast pozostałe serwisy odpowiedzialne za logikę biznesową
działają rzeczywiście. Takie podejście umożliwia sprawdzenie zarówno poprawności odpowiedzi HTTP,
walidacji danych, jak i integracji kontrolerów z warstwą serwisów.
Testy pozwalają na wykrywanie błędów na poziomie integracji, jednocześnie gwarantując stabilność
i powtarzalność testów poprzez eliminację zależności od zewnętrznych systemów.


\subsection{Wyniki testów, napotkane błędy oraz zastosowane rozwiązania}



\end{document}