\documentclass[../../spr.tex]{subfiles}

\begin{document}
\section{Wprowadzenie/Cel laboratorium}

\subsection{Krótki opis aplikacji}
Aplikacja to system do zarządzania siłownią.
Umożliwia kompleksowe zarządzanie obiektem fitness,
w tym: rejestrację użytkowników, zarządzanie członkostwami,
planowanie sesji treningowych oraz obsługę płatności.
System zapewnia bezpieczną autoryzację poprzez integrację z Keycloak (OIDC)
i oferuje REST API z pełną dokumentacją w Swaggerze.
\subsection{Wykorzystane technologie oraz narzędzia}


\begin{itemize}
  \item \textbf{Nix} - System do zarządzania pakietami i środowiskami – pozwala tworzyć powtarzalne konfiguracje projektów.

  \item \textbf{git} - system kontroli wersji

  \item \textbf{GitHub Actions} -
        System CI/CD w GitHubie do automatyzacji testów, buildów i wdrożeń przy pomocy workflowów.

  \item \textbf{Spring Boot} -
        Framework w Javie do szybkiego tworzenia aplikacji webowych i mikroserwisów.

  \item \textbf{Swagger (OpenAPI)} -
        Narzędzie do generowania i testowania dokumentacji REST API w sposób interaktywny.

  \item \textbf{OIDC} -
        Protokół uwierzytelniania oparty na OAuth 2.0 – pozwala na bezpieczne logowanie użytkowników.

  \item \textbf{Keycloak} -
        Open-source’owy serwer tożsamości z obsługą SSO, OIDC i integracją z LDAP.

  \item \textbf{PostgreSQL} -
        Zaawansowany relacyjny system baz danych, otwartoźródłowy, skalowalny i wydajny.

  \item \textbf{Stripe} -
        Platforma płatnicza do obsługi płatności online – łatwa integracja przez REST API.

  \item \textbf{Taskfile} -
        Lekki zamiennik Makefile – automatyzuje zadania developerskie w plikach YAML.

  \item \textbf{adminer} - system zarządzania bazą danych

  \item \textbf{lombok} - redukcja kodu typu boilerplate

  \item \textbf{docker-compose} - do jednolitego środowiska uruchominiowego (baza danych, keycloak, stripe)

  \item
\end{itemize}



\end{document}